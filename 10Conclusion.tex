% Section 8 : Conclusion

\section{Conclusion}

% =====================================================
\begin{frame}{Récapitulatif}
  
  Nous avons fait un \textbf{voyage complet} dans l'univers de l'architecture logicielle~!
  
  \vspace{0.5cm}
  
  \textbf{Ce que nous avons appris~:}
  
  \begin{enumerate}
    \item \textbf{L'Architecture Logicielle} = Comment organiser le code d'une application
    \item \textbf{Les Principes Fondamentaux}~: Abstraction, Encapsulation, Séparation des responsabilités
    \item \textbf{La Démarche en 4~Étapes}~: Analyser → Définir → Concevoir → Valider
    \item \textbf{Les 2~Vues Essentielles}~: Couches (logique) et Niveaux (physique)
    \item \textbf{Les Types d'Architectures}~: Monolithique, Microservices, SOA, Hexagonale...
    \item \textbf{La Cartographie}~: Diagrammes de Packages, Composants, Déploiement
  \end{enumerate}

\end{frame}

% =====================================================
\begin{frame}{Les Points Clés à Retenir (1/2)}
  
  \textbf{1. Il n'y a pas UNE seule bonne architecture}
  
  \begin{itemize}
    \item Chaque architecture a ses avantages et inconvénients
    \item Le choix dépend~: de la taille de l'application, de l'équipe, des besoins de performance, du budget, du temps
  \end{itemize}
  
  \vspace{0.3cm}
  
  \textbf{2. On peut combiner plusieurs architectures}
  
  \begin{itemize}
    \item Architecture en Couches (organisation du code)
    \item + Pattern MVC (pour la présentation)
    \item + Architecture Hexagonale (pour isoler le métier)
    \item + Déploiement N-Tiers (pour la scalabilité)
  \end{itemize}

\end{frame}

% =====================================================
\begin{frame}{Les Points Clés à Retenir (2/2)}
  
  \textbf{3. L'architecture évolue avec le temps}
  
  \begin{itemize}
    \item On peut commencer simple et évoluer~:
    \item Monolithique → Monolithique Modulaire → Microservices
  \end{itemize}
  
  \vspace{0.5cm}
  
  \textbf{4. La qualité de l'architecture est cruciale}
  
  \begin{columns}[T]
    \begin{column}{0.48\textwidth}
      \begin{exampleblock}{Bonne architecture}
        \begin{itemize}
          \item [\checkmark] Code facile à comprendre
          \item [\checkmark] Code facile à modifier
          \item [\checkmark] Code facile à tester
          \item [\checkmark] Application performante
          \item [\checkmark] Application sécurisée
        \end{itemize}
      \end{exampleblock}
    \end{column}
    
    \begin{column}{0.48\textwidth}
      \begin{alertblock}{Mauvaise architecture}
        \begin{itemize}
          \item [\texttimes] Code spaghetti
          \item [\texttimes] Bugs en cascade
          \item [\texttimes] Impossible d'ajouter des fonctionnalités
          \item [\texttimes] Équipe frustrée
        \end{itemize}
      \end{alertblock}
    \end{column}
  \end{columns}

\end{frame}

% =====================================================
\begin{frame}{Aller Plus Loin}
  
  \textbf{Pour les Débutants~:}
  \begin{itemize}
    \item Bien comprendre les \textbf{couches} (Présentation, Métier, Données)
    \item Apprendre le \textbf{pattern MVC}
    \item Pratiquer la \textbf{séparation des responsabilités}
  \end{itemize}
  
  \vspace{0.3cm}
  
  \textbf{Pour les Intermédiaires~:}
  \begin{itemize}
    \item Approfondir l'\textbf{architecture en Couches} (5~couches)
    \item Maîtriser les \textbf{Design Patterns} (Repository, Singleton, Observer...)
    \item Découvrir l'\textbf{architecture Hexagonale}
  \end{itemize}
  
  \vspace{0.3cm}
  
  \textbf{Pour les Avancés~:}
  \begin{itemize}
    \item Explorer les \textbf{Microservices}
    \item Approfondir le \textbf{Domain-Driven Design (DDD)}
    \item Maîtriser l'\textbf{Event-Driven Architecture}
  \end{itemize}

\end{frame}

% =====================================================
\begin{frame}{Ressources pour Continuer}
  
  \textbf{Livres~:}
  
  \begin{itemize}
    \item \og{}Clean Architecture\fg{} - Robert C. Martin
    \item \og{}Domain-Driven Design\fg{} - Eric Evans
    \item \og{}Building Microservices\fg{} - Sam Newman
    \item \og{}Patterns of Enterprise Application Architecture\fg{} - Martin Fowler
  \end{itemize}
  
  \vspace{0.3cm}
  
  \textbf{Sites Web~:}
  
  \begin{itemize}
    \item \url{https://martinfowler.com/} (articles sur l'architecture)
    \item \url{https://microservices.io/} (patterns microservices)
    \item \url{https://c4model.com/} (modélisation d'architecture)
  \end{itemize}
  
  \vspace{0.3cm}
  
  \textbf{Pratique~:}
  
  \begin{itemize}
    \item Créez de petits projets
    \item Refactorez du code existant
    \item Lisez du code open source
    \item Participez à des code reviews
  \end{itemize}

\end{frame}

% =====================================================
\begin{frame}{Mot de la Fin}
  
  \begin{alertblock}{L'architecture logicielle, c'est comme construire une maison}
    \begin{itemize}
      \item On ne met pas le toit avant les fondations
      \item On planifie avant de construire
      \item On peut rénover et améliorer avec le temps
      \item Une bonne architecture dure des années
    \end{itemize}
  \end{alertblock}
  
  \vspace{0.5cm}
  
  \begin{block}{Mais surtout}
    \centering
    \textit{\og{}La meilleure architecture est celle qui répond aux besoins actuels tout en permettant l'évolution future\fg{}}
  \end{block}

\end{frame}

% =====================================================
\begin{frame}{N'oubliez pas}
  
  \begin{itemize}
    \item \textbf{Commencez simple} (KISS~: Keep It Simple, Stupid)
    \item \textbf{N'ajoutez de la complexité que quand c'est nécessaire} (YAGNI~: You Ain't Gonna Need It)
    \item \textbf{Refactorez régulièrement} (amélioration continue)
  \end{itemize}

\end{frame}

% =====================================================
\begin{frame}{Mission Accomplie}
  
  \textbf{Vous savez maintenant~:}
  
  \begin{itemize}
    \item [\checkmark] Ce qu'est l'architecture logicielle
    \item [\checkmark] Pourquoi c'est important
    \item [\checkmark] Comment la concevoir
    \item [\checkmark] Quels types d'architectures existent
    \item [\checkmark] Comment la cartographier
  \end{itemize}
  
  \vspace{1cm}
  
  \begin{center}
    \Large \textbf{\textcolor{IFSPAccent}{Félicitations~!}}
    
    \vspace{0.3cm}
    
    \large Vous êtes maintenant capable de comprendre et de discuter d'architecture logicielle comme un pro~!
  \end{center}

\end{frame}

% =====================================================
\begin{frame}{Annexe~: Glossaire des Termes Principaux}
  
  \begin{center}
    \small
    \begin{tabular}{|l|p{7cm}|}
      \hline
      \textbf{Terme} & \textbf{Définition Simple} \\
      \hline
      Abstraction & Cacher les détails compliqués \\
      \hline
      API & Interface pour communiquer avec un système \\
      \hline
      Couplage & Degré de dépendance entre modules \\
      \hline
      Cohésion & Degré de liaison entre éléments d'un module \\
      \hline
      Composant & Morceau de code réutilisable \\
      \hline
      Design Pattern & Recette pour résoudre un problème courant \\
      \hline
      Framework & Boîte à outils de développement \\
      \hline
      Microservices & Architecture avec plein de petits services indépendants \\
      \hline
      Monolithe & Application d'un seul bloc \\
      \hline
      Repository & Gestionnaire d'accès aux données \\
      \hline
      Scalabilité & Capacité à gérer plus de charge \\
      \hline
    \end{tabular}
  \end{center}

\end{frame}

% =====================================================
\begin{frame}[plain]
  \centering
  \Huge \textbf{\textcolor{IFSPPrimario}{Merci pour votre attention~!}}
  
  \vspace{1cm}
  
  \Large Questions~?
\end{frame}
