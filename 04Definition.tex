% Section 2 : Définition de l'Architecture Logicielle

\section{Définition de l'Architecture Logicielle}

% =====================================================
\begin{frame}{Définition Simple}
  
  \vspace{1cm}
  \begin{block}{L'Architecture Logicielle, c'est~:}
    \centering
    \textbf{\Large La façon dont on organise et on structure le code d'une application pour qu'elle fonctionne bien}
  \end{block}

\end{frame}

% =====================================================
\begin{frame}{Les Deux Rôles de l'Architecture Logicielle}
  
  \begin{columns}[T]
    \begin{column}{0.48\textwidth}
      \begin{block}{Rôle n°1~: Structurer}
        Elle organise les solutions \textbf{les mieux adaptées} pour répondre aux besoins de l'utilisateur.
        
        \vspace{0.3cm}
        
        \textbf{Exemple~:} Pour une application e-commerce~:
        \begin{itemize}
          \item Où mettre le code du panier~?
          \item Où mettre le calcul des prix~?
          \item Où mettre l'affichage~?
        \end{itemize}
      \end{block}
    \end{column}
    
    \begin{column}{0.48\textwidth}
      \begin{block}{Rôle n°2~: Décomposer}
        Elle découpe l'application en petits morceaux faciles à comprendre et à modifier.
        
        \vspace{0.3cm}
        
        \textbf{Analogie~:} Comme un puzzle géant~!
        \begin{itemize}
          \item Chaque pièce = un morceau de code
          \item L'architecture = les règles d'assemblage
        \end{itemize}
      \end{block}
    \end{column}
  \end{columns}

\end{frame}

% =====================================================
\begin{frame}{Les 5~Concepts Clés de Décomposition (1/3)}
  
  \textbf{1. Les Couches (Layers)}
  
  \begin{itemize}
    \item Comme un gâteau à étages
    \item Chaque étage = une couche avec un rôle précis
    \item \textbf{Exemple~:} Couche interface → Couche logique → Couche données
  \end{itemize}
  
  \vspace{0.5cm}
  
  \textbf{2. Les Modules}
  
  \begin{itemize}
    \item Un gros dossier qui regroupe des fonctionnalités liées
    \item \textbf{Exemple~:} Module \og{}Gestion des Clients\fg{}, Module \og{}Gestion des Commandes\fg{}
  \end{itemize}

\end{frame}

% =====================================================
\begin{frame}{Les 5~Concepts Clés de Décomposition (2/3)}
  
  \textbf{3. Les Composants}
  
  \begin{itemize}
    \item Un morceau de code réutilisable, comme une brique de LEGO
    \item \textbf{Exemple~:} Un bouton \og{}Valider\fg{}, un formulaire de connexion
  \end{itemize}
  
  \vspace{0.5cm}
  
  \textbf{4. Les Design Patterns}
  
  \begin{itemize}
    \item Une recette de cuisine pour résoudre un problème courant
    \item \textbf{Exemple~:} 
    \begin{itemize}
      \item Pattern \og{}Singleton\fg{}~: une seule connexion à la BD
      \item Pattern \og{}Observer\fg{}~: prévenir tous les intéressés quand quelque chose change
    \end{itemize}
  \end{itemize}

\end{frame}

% =====================================================
\begin{frame}{Les 5~Concepts Clés de Décomposition (3/3)}
  
  \textbf{5. Les Frameworks}
  
  \begin{itemize}
    \item Une boîte à outils toute prête avec plein de fonctions déjà codées
    \item \textbf{Exemples~:}
    \begin{itemize}
      \item React (pour faire des interfaces web)
      \item Spring (pour faire des applications Java)
      \item Django (pour faire des applications Python)
    \end{itemize}
  \end{itemize}

\end{frame}
