% Section 1 : Introduction - Entrer dans la Boîte Blanche

\section{Introduction~: Entrer dans la Boîte Blanche}

% =====================================================
\begin{frame}{Pourquoi cette présentation~?}
  
  \textbf{Imaginez que vous avez construit une maison.} Jusqu'à maintenant, nous avons parlé de~:
  
  \begin{itemize}
    \item \textbf{L'architecture applicative}~: c'est comme le plan des pièces de votre maison (cuisine, salon, chambres...)
    \item \textbf{L'architecture technique}~: ce sont les matériaux (briques, béton, tuyaux, électricité...)
  \end{itemize}
  
  \vspace{0.5cm}
  
  \begin{alertblock}{Maintenant}
    Nous allons découvrir \textbf{ce qui se passe À L'INTÉRIEUR de chaque pièce}~!
  \end{alertblock}

\end{frame}

% =====================================================
\begin{frame}{Qu'est-ce que la \og{}Boîte Blanche\fg{}~?}
  
  \begin{columns}[T]
    \begin{column}{0.48\textwidth}
      \begin{block}{Boîte Noire}
        \begin{itemize}
          \item Vue de l'extérieur
          \item On ne voit pas comment ça fonctionne
          \item On voit juste que ça marche
        \end{itemize}
      \end{block}
    \end{column}
    
    \begin{column}{0.48\textwidth}
      \begin{alertblock}{Boîte Blanche}
        \begin{itemize}
          \item Vue de l'intérieur
          \item On découvre les composants
          \item On comprend l'organisation
          \item On voit les communications
        \end{itemize}
      \end{alertblock}
    \end{column}
  \end{columns}
  
  \vspace{0.3cm}
  
  \textbf{Analogie}~: C'est comme ouvrir le capot d'une voiture pour voir le moteur~!

\end{frame}

% =====================================================
\begin{frame}{Le Couplage Fort entre Architecture Logicielle et Technique}
  
  \begin{alertblock}{Important}
    Les architectures logicielles et techniques sont en \textbf{\og{}couplage fort\fg{}}.
  \end{alertblock}
  
  \vspace{0.3cm}
  
  \textbf{Qu'est-ce que ça veut dire~?}
  
  \begin{itemize}
    \item Imaginez deux danseurs qui dansent un tango~:
    \begin{itemize}
      \item Si l'un fait un pas à gauche, l'autre \alert{DOIT suivre}
      \item Les choix de l'un influencent directement l'autre
      \item Ils sont \textbf{couplés}~: ils doivent bouger ensemble~!
    \end{itemize}
  \end{itemize}
  
  \vspace{0.3cm}
  
  \textbf{De la même façon}~:
  \begin{itemize}
    \item Les choix pour l'architecture \textbf{logicielle} (comment organiser le code)
    \item Influencent directement les choix pour l'architecture \textbf{technique} (serveurs, base de données...)
    \item \alert{Et vice-versa~!}
  \end{itemize}

\end{frame}

% =====================================================
\begin{frame}{Les 3~Contraintes Essentielles}
  
  En plus de faire fonctionner l'application, l'architecture logicielle doit garantir~:
  
  \vspace{0.5cm}
  
  \begin{enumerate}
    \item \textbf{\textcolor{IFSPAccent}{Sécurité}}
    \begin{itemize}
      \item Comme mettre une serrure sur votre porte
      \item Empêcher les méchants d'entrer
      \item Protéger vos données
    \end{itemize}
    
    \vspace{0.3cm}
    
    \item \textbf{\textcolor{IFSPAccent}{Disponibilité}}
    \begin{itemize}
      \item Comme avoir de l'électricité 24h/24
      \item L'application doit fonctionner tout le temps
    \end{itemize}
    
    \vspace{0.3cm}
    
    \item \textbf{\textcolor{IFSPAccent}{Performance}}
    \begin{itemize}
      \item Comme avoir une voiture rapide
      \item Répondre vite, même avec beaucoup d'utilisateurs
    \end{itemize}
  \end{enumerate}

\end{frame}
