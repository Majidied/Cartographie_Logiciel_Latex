% Section 3 : Démarche en 4 Étapes

\section{Démarche en 4~Étapes}

% =====================================================
\begin{frame}{Démarche en 4~Étapes (schéma)}

  \begin{center}
    \begin{tikzpicture}[
      node distance=0.7cm,
      every node/.style={font=\bfseries},
      step/.style={rounded corners, minimum width=7.8cm, minimum height=0.8cm, align=center, fill=none},
      arrow/.style={->, thick}
    ]
      \node[rounded corners, minimum width=7.8cm, minimum height=0.8cm, align=center, fill=IFSPPrimario!25] (s1) {1. Analyser les besoins};
      \node[rounded corners, minimum width=7.8cm, minimum height=0.8cm, align=center, fill=IFSPAccent!20, below=of s1] (s2) {2. Définir la structure};
      \node[rounded corners, minimum width=7.8cm, minimum height=0.8cm, align=center, fill=IFSPPrimario!25, below=of s2] (s3) {3. Concevoir les composants};
      \node[rounded corners, minimum width=7.8cm, minimum height=0.8cm, align=center, fill=IFSPAccent!20, below=of s3] (s4) {4. Valider \& ajuster};

      \draw[arrow] (s1) -- (s2);
      \draw[arrow] (s2) -- (s3);
      \draw[arrow] (s3) -- (s4);

      \draw[arrow, dashed] ([xshift=0.2cm]s4.east) .. controls +(0.6,0) and +(0.6,0) .. ([xshift=0.2cm]s1.east);
      \node[font=\small] at ([xshift=1.3cm,yshift=-1.5cm]s2.east) {itératif};
    \end{tikzpicture}
  \end{center}

  \vspace{0.1cm}
  \centering
  \small Itératif (boucle) \textbullet{} incrémental (ajouts progressifs)

\end{frame}

