% Section 3 : Démarche en 4 Étapes

\section{Démarche en 4~Étapes}

% =====================================================
\begin{frame}{Comment Créer une Architecture Logicielle~?}
  
  \begin{alertblock}{Principe}
    Créer une architecture logicielle, c'est comme construire une maison~: on ne commence pas par le toit~!
  \end{alertblock}
  
  \vspace{0.3cm}
  
  Il faut suivre \textbf{4~étapes}, et on les répète plusieurs fois (\textbf{de manière itérative et incrémentale}).
  
  \vspace{0.5cm}
  
  \textbf{C'est quoi \og{}itérative et incrémentale\fg{}~?}
  
  \begin{itemize}
    \item \textbf{Itérative} = on répète les étapes plusieurs fois, en améliorant à chaque fois
    \item \textbf{Incrémentale} = on ajoute des morceaux petit à petit, comme des couches de peinture
  \end{itemize}

\end{frame}

% =====================================================
\begin{frame}{ÉTAPE 1~: Analyser les Besoins}
  
  \textbf{But~:} Comprendre ce que l'application doit faire.
  
  \vspace{0.3cm}
  
  \textbf{Questions à se poser~:}
  
  \begin{itemize}
    \item Qui va utiliser l'application~? (des clients~? des employés~?)
    \item Qu'est-ce que l'application doit faire~? (vendre des produits~? gérer des stocks~?)
    \item Combien de personnes vont l'utiliser~? (10~? 1000~? 1~million~?)
    \item Quelles sont les contraintes~? (budget~? délai~? sécurité~?)
  \end{itemize}
  
  \vspace{0.3cm}
  
  \begin{exampleblock}{Exemple concret~: Application e-commerce}
    \begin{itemize}
      \item Les clients doivent pouvoir chercher des produits
      \item Les clients doivent pouvoir payer en ligne
      \item L'application doit supporter 10\,000 clients simultanés
      \item Les paiements doivent être ultra-sécurisés
    \end{itemize}
  \end{exampleblock}

\end{frame}

% =====================================================
\begin{frame}{ÉTAPE 2~: Définir la Structure}
  
  \textbf{But~:} Choisir comment organiser l'application.
  
  \vspace{0.3cm}
  
  \textbf{Décisions à prendre~:}
  
  \begin{enumerate}
    \item \textbf{Quel type d'architecture choisir~?}
    \begin{itemize}
      \item Monolithique~? (tout en un seul bloc)
      \item Microservices~? (plein de petits services indépendants)
      \item Client-Serveur~? (un client qui parle à un serveur)
    \end{itemize}
    
    \item \textbf{Combien de couches~?}
    \begin{itemize}
      \item 3~couches~? (Présentation, Logique, Données)
      \item 5~couches~? (encore plus de séparation)
    \end{itemize}
    
    \item \textbf{Quels frameworks utiliser~?}
    \begin{itemize}
      \item React pour l'interface~?
      \item Spring Boot pour le backend~?
    \end{itemize}
  \end{enumerate}

\end{frame}

% =====================================================
\begin{frame}{ÉTAPE 2~: Définir la Structure (Analogie)}
  
  \begin{block}{Analogie}
    C'est comme décider si votre maison sera~:
    
    \begin{itemize}
      \item Une maison individuelle (monolithique)
      \item Un immeuble avec plusieurs appartements (microservices)
      \item Le nombre d'étages (les couches)
    \end{itemize}
  \end{block}

\end{frame}

% =====================================================
\begin{frame}{ÉTAPE 3~: Concevoir les Composants (1/2)}
  
  \textbf{But~:} Créer les briques de base de l'application.
  
  \vspace{0.3cm}
  
  \textbf{Ce qu'on fait~:}
  
  \begin{enumerate}
    \item \textbf{Identifier les composants nécessaires}
    \begin{itemize}
      \item Composant \og{}Liste des produits\fg{}
      \item Composant \og{}Panier d'achat\fg{}
      \item Composant \og{}Formulaire de paiement\fg{}
    \end{itemize}
    
    \item \textbf{Définir comment ils communiquent}
    \begin{itemize}
      \item Le panier envoie les infos au paiement
      \item Le paiement vérifie avec la banque
      \item La banque renvoie OK ou KO
    \end{itemize}
  \end{enumerate}

\end{frame}

% =====================================================
\begin{frame}{ÉTAPE 3~: Concevoir les Composants (2/2)}
  
  \begin{enumerate}
    \setcounter{enumi}{2}
    \item \textbf{Choisir les design patterns}
    \begin{itemize}
      \item Pattern MVC pour séparer l'affichage de la logique
      \item Pattern Repository pour accéder aux données
    \end{itemize}
  \end{enumerate}
  
  \vspace{0.5cm}
  
  \begin{block}{Analogie}
    C'est comme décider~:
    \begin{itemize}
      \item Quels meubles mettre dans chaque pièce
      \item Où placer les prises électriques
      \item Comment les pièces sont connectées
    \end{itemize}
  \end{block}

\end{frame}

% =====================================================
\begin{frame}{ÉTAPE 4~: Valider et Ajuster}
  
  \textbf{But~:} Vérifier que l'architecture fonctionne bien.
  
  \vspace{0.3cm}
  
  \textbf{Ce qu'on vérifie~:}
  
  \begin{enumerate}
    \item \textbf{La qualité}
    \begin{itemize}
      \item Est-ce que le code est propre et bien organisé~?
      \item Est-ce qu'on peut facilement ajouter de nouvelles fonctionnalités~?
    \end{itemize}
    
    \item \textbf{Les performances}
    \begin{itemize}
      \item Est-ce que l'application est rapide~?
      \item Est-ce qu'elle peut gérer beaucoup d'utilisateurs~?
    \end{itemize}
    
    \item \textbf{La sécurité}
    \begin{itemize}
      \item Est-ce que les données sont protégées~?
      \item Est-ce qu'on peut pirater l'application~?
    \end{itemize}
  \end{enumerate}
  
  \vspace{0.3cm}
  
  \begin{alertblock}{Si ça ne va pas}
    On retourne à l'étape 2~ou 3~et on améliore~!
  \end{alertblock}

\end{frame}

% =====================================================
\begin{frame}{Pourquoi \og{}Itératif et Incrémental\fg{}~?}
  
  \begin{alertblock}{Parce qu'on ne peut pas tout prévoir~!}
    Au début du projet, on ne sait pas tout. Donc~:
  \end{alertblock}
  
  \vspace{0.3cm}
  
  \begin{itemize}
    \item On fait une première version simple (Itération 1)
    \item On la teste
    \item On l'améliore (Itération 2)
    \item On ajoute des fonctionnalités (Incrémental)
    \item On teste encore
    \item Et ainsi de suite...
  \end{itemize}
  
  \vspace{0.5cm}
  
  \begin{exampleblock}{Exemple}
    \begin{itemize}
      \item \textbf{Itération 1}~: Afficher une liste de produits
      \item \textbf{Itération 2}~: Ajouter un panier
      \item \textbf{Itération 3}~: Ajouter le paiement
      \item \textbf{Itération 4}~: Ajouter la gestion des comptes clients
    \end{itemize}
    
    Chaque itération = une version de plus en plus complète~!
  \end{exampleblock}

\end{frame}
